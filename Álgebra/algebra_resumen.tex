\documentclass{article}
\usepackage{graphicx} % Required for inserting images
\usepackage[spanish, english]{babel}
\usepackage{fullpage}
\usepackage{amsmath, amsfonts}

% Títulos
\title{Matrices Algebra}
\author{Ian Chen}
\date{6 de Junio, 2024}

% Documento
\begin{document}
\maketitle

%-------------------------------------------------%
\section{Sistemas de ecuaciones lineales}
\subsection{Definición}
Un sistema de ecuaciones 'lineales' son ecuaciones 'lineales'\footnote{Ecuaciones o polinomios de grado 1} que comparten misma solución. \\[6pt]
Vectorialmente se podría pensar como la intersección entre un conjunto de planos. Pensando de esta manera podríamos deducir que las posibles soluciones son un punto, todos o ninguno.\\[6pt]
Se tiene el siguiente sistema lineal de m-ecuaciones lineales con n-incógnitos por definición:

\[\left\{
\begin{array}{ccccccccc}
    a_{11}x_1 & + & a_{12}x_2 & + & \dots & + & a_{in}x_n & = & b_1 \\
    a_{21}x_1 & + & \dots & + & a_{2n}x_n & = & b_2 \\
    \vdots \\
    a_{m1}x_1 & + & a_{m2}x_2 & + & \dots & + & a_{mn}x_n & = & b_m
\end{array}
\right.
\]
\noindent
Tipos de solución:
\begin{itemize}
    \item Tiene Solución → sistema compatible
    \begin{itemize}
        \item Solución única (Sistema compatible determinado)
        \item Solución infinitas (Sistema compatible indeterminado)
    \end{itemize}
    \item No tiene solución → sistema incompatible
\end{itemize}

\subsection{Pasos de solución}
\begin{itemize}
    \item Cambiar el orden de las ecuaciones
    \item Cambiar una ecuación por un múltiplo no nulo de ella
        $$E_i \rightarrow kE_i (k \neq 0)$$
    \item Cambiar una ecuación por dicha ecuación + un múltiplo de cualquier otra 
        $$E_i \rightarrow E_i + kE_j (i \neq j)$$
    \item Utilizando las propiedades, pasarlo a matrices y llegar a un sistema escalonado.
\end{itemize}

\subsection{Ejemplo}
Teniendo el siguiente sistema lineal se puede representarlo como una matriz para facilitar su solución:
% sistema de ec
\[\left\{
\begin{array}{ccccccc}
    3x & + & 4y & - & 2z & = & 5 \\
    -x & + & y & - & z & = & 3
\end{array}
\right.
\]

% Matrices
\[
\left(
\begin{array}{ccc|c}
     3 & 4 & -2 & 5 \\
     -1 & 1 & 1 & 3
\end{array}
\right) 
\, F_2 \rightarrow 3F_2 + F_1 \,
\left(
\begin{array}{ccc|c}
     3 & 4 & -2 & 5 \\
     0 & 7 & 1 & 14
\end{array}
\right) 
\]
Luego se obtiene que $$7y+z=14 \rightarrow z=14-7y$$ Teniendo esta identidad se puede despejar el resto de las ecuaciones en base a ella: 
\begin{align*}
    3x+4y-28+14y & = 5\\
    3x+18y &=33 \\
    3x &=33-18y\\
    x &= 11-6y
\end{align*}
Finalmente se obtiene la siguiente solución
\[S:
\left\{
\begin{array}{cc}
     x: & 11-6y \\
     y: & y \\
     z: & 14-7y
\end{array}
\right.
\]
$$S=\{(11-6y, y, 14-7y) \, /  \,y \in \mathbb{R}\} = \{(11, 0, 14) + (-6y, y, -7y)\}$$
$$\Vec{X} = (11, 0, 14) + \lambda(-6, 1, 7)$$

\vspace{1cm}
%---------------------------------------%
\section{Matrices}
\subsection{Definición}
\begin{itemize}
    \item La matriz es una tabla de números
    \item Sirve para organizar información (principalmente)
    \item Ejemplo de compañía telefónica, vuelos, etc
    \item La matriz es un arreglo (array)
\end{itemize}

% SUMA 
\subsection{Suma de matrices}
\subsubsection{Definición}
\subsubsection{propiedades}

% Multiplicación
\subsection{Multiplicación de matrices}
\subsubsection{Definición}
\subsubsection{propiedades}

% Identidad
\subsection{Matriz identidad}
\subsubsection{Definición}
\subsubsection{propiedades}

% Trasposición
\subsection{Trasposición}
\subsubsection{Definición}
\subsubsection{propiedades}

% Matriz cuadrada
\subsection{Matriz cuadrada}
\subsubsection{Definición}
\subsubsection{propiedades}

% Inversa
\subsection{Matriz inversa}
\subsubsection{Definición}
\subsubsection{propiedades}

% Determinantes
\subsection{Determinantes}
\subsubsection{Definición}
\subsubsection{propiedades}

%-----------------------------------%
\end{document}
