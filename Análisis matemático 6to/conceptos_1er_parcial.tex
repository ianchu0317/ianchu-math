\documentclass{article}
\usepackage{graphicx} % Required for inserting images
\usepackage{amsfonts, amsmath, amssymb}
\usepackage{hyperref}
\hypersetup{
    colorlinks=true,
    linkcolor=blue,
    filecolor=magenta,      
    urlcolor=cyan,
    pdftitle={Notas - propiedades y teoremas},
    pdfpagemode=FullScreen,
}
\usepackage{fullpage}

%---------------------
\title{Propiedades y Teoremas - Primer parcial}
\author{Ian Chen}
\date{June 2024}

%--------------------
\begin{document}
% \maketitle
\section{Sucesiones}
\subsection{Subsucesiones}
Una subsucesión de una sucesión $\displaystyle a_n$ es una sucesión que está dentro de la sucesión $a_n$.
Una subsucesión por ejemplo nos brinda información acerca de la sucesión la cual pertenece. 

Por ejemplo la sucesión de números pares $a_n = \{2n ,\, n \in \mathbb{N} \} = \{2, 4, 6, 8, 10 \dots \}$ podría tener de subsucesión $a_{n_k} = \{2^k, k \in \mathbb{N}\} = \{2, 4, 8, 16\dots \}$ que contenga las potencias de 2, ya que las potencias de 2 también son números pares.

Algunas propiedades importantes de subsucesiones son:
\begin{itemize}
    \item[-] Si $a_n$ converge a $L \neq \infty$, entonces todas las subsucesiones también convergen a $L$
    \item[-] Si $a_n$ es una sucesión acotada, entonces todas las subsucesiones también son acotadas
    \item[-] Si las subsucesiones tienden a límites distintos, entonces la sucesión $a_n$ no converge (osea que no está acotado $\displaystyle \lim \limits_{x \to \infty}a_n=\infty$)
\end{itemize}
Información extraída de la siguiente \href{https://www.matesfacil.com/BAC/progresiones/subsucesion-sucesion-parcial-convergente-propiedades-ejemplos-limites-problemas-resueltos.html}{página}.

\vspace{0.5cm}
%-----------------------
\subsection{Factoriales}
Ecuaciones importantes de factoriales:
\begin{itemize}
    \item[-] Se define $0\,! = 1$
    \item[-] Ecuaciones importantes dadas por despeje 
    \begin{align*}
        n\,! &= 1 \cdot 2 \cdot 3 \dots n \\[6pt]
        n\,! \cdot (n+1) &= (n+1)\,! \\[6pt]
        \frac{(n+1)\,!}{(n+1)} &= n\,! \\[6pt]
        \frac{(n+1)\,!}{n\,!} &= n+1
    \end{align*}
\end{itemize}


%-----------------------
\subsection{Criterio de D'Alambert y Cauchy}
Los dos criterios sirven para averiguar si la sucesión tiende a $0$ o a infinito.
D'Alambert (criterio del cociente) dice que si $a_n \neq 0$, entonces $\lim \limits_{n \to \infty} \dfrac{a_{n+1}}{a_n} = L$, mientras que Cauchy (criterio de raiz enésima) dice que $\displaystyle \lim \limits_{n \to \infty} \sqrt[n]{a_n} = L$.

Luego, 
\[
\lim \limits_{x \to \infty} a_n \left\{
\begin{array}{lcr}
    \infty & si & \,L>1 \\
     0 & si & \,0 \leq L \leq 1 \\
     indet & si & \,L=1
\end{array}
\right.
\]

%--------------------
\section{Límites}


%--------------------
\section{Derivadas}


\end{document}
